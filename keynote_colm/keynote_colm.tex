\documentclass[a4paper]{article}
\usepackage{interspeech2011}
\setcounter{page}{1}
\sloppy		% better line breaks
\ninept

\title{From Actor To Avatar: Real world challenges in capturing the human face}

\makeatletter
\def\name#1{\gdef\@name{#1\\}}
\makeatother
\name{{\em Colm Massey}}

\address{Head of research and development at AudioMotion, Oxford, UK}

\begin{document}
\maketitle

\begin{abstract}
In Nov 2009, James Cameron's seminal movie Avatar raised the bar for what can be achieved with facial animation. For many, the ``Uncanny Valley'', the bane of the animation industry for so long seemed to have been comprehensively bridged. But what does it take to create such jaw dropping animations, and is it possible to create them on anything less than a titanic budget with armies of animators? This talk will provide an overview of what it takes to create high end facial animation using motion capture technology. What are the current challenges facing CG studios, attempting to map human performances onto computer game and movie characters? What tools and pipelines are currently used? Can these tools be of use to AVSP researchers? What can AVSP researchers contribute to the goal of making high end facial animation accessible to those of us with shallower pockets?
\end{abstract}
\end{document}

% do not remove (emacs configuration)
% Local variables:
% enable-local-variables: t
% ispell-local-dictionary: "en_GB"
% mode: flyspell
% eval: (flyspell-buffer)
% End:
