\documentclass[a4paper]{article}
\usepackage{interspeech2011}
\setcounter{page}{1}
\sloppy		% better line breaks
\ninept

\title{Acoustical and visual processing in the animal kingdom}
\author{Sverre Sj{\"o}lander}

\address{Professor Emeritus in Zoology, Linköping University, Sweden}

\begin{document}
\maketitle
%
\begin{abstract}
First, it should be pointed out that animals in general only are sensitive to those parts of the available information that is of importance to their species. If the only acoustical signal of importance to a grasshopper is the sound of the male, this is all the female can hear. The only molecules that an animal reacts to are those of any significance to the species --- they are ``smell''. If colours are of importance, colour vision will develop, but only then. I.e. the information every species receives is the one that evolution has selected as important.

Secondly, in the animal kingdom, the question is rarely of some kind of information corresponds to the actual situation in reality. In sending information, whether acoustical, visual, tactical, olfactory, electrical or whatever, the behaviour is directed at creating something favourable for the sender. You do not want to tell the truth about e.g. your strength or your muscular mass: you want the receiver to be impressed, as much as possible. Consequently the animal world is full of bluffing. You increase your body size as much as possible, by inflating, by special feathers or fins, etc. You deepen your voice by e.g. using a big balloon under you jaw, like in frogs, thus sounding bigger than you are.

The receiver on his hand has to try to look behind the bluff, to estimate how strong or dangerous the sender actually is. Thus, in animal communication we get a series from bluffing to trying to test the bluff, and ultimately to fight, if nobody gives up. In short, it is just like human party conversation: try to give a favourable impression of yourself, exaggerate your good points and cover up the weak ones, being aware that the receiver the whole time deducts from what she or he receives, to arrive at a truthful impression. Too much boasting is counterproductive, since the receiver will deduct even more.

A very important point is that humans, and some degree the great apes (at least) --- probably dolphins as well --- are aware of their own bluffs and lying, or exaggerations. Other animals do not seem to understand what they are doing, in this sense. But a chimpanzee may produce a simple lie, to get some advantage. It is of interest that human children do not start to lie, more consciously, until about 4 years of age.
\end{abstract}
\end{document}

% do not remove (emacs configuration)
% Local variables:
% enable-local-variables: t
% ispell-local-dictionary: "en_GB"
% mode: flyspell
% eval: (flyspell-buffer)
% End:
