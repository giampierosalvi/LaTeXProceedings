% Copyright 2017, Giampiero Salvi <giampi@kth.se>
% electronic version of the proceedings
\documentclass{confproc}
\usepackage{hyperref}
\hypersetup{
%    bookmarks=true,         % show bookmarks bar?
%    unicode=false,          % non-Latin characters in Acrobat’s bookmarks
%    pdftoolbar=true,        % show Acrobat’s toolbar?
%    pdfmenubar=true,        % show Acrobat’s menu?
%    pdffitwindow=false,     % window fit to page when opened
%    pdfstartview={FitH},    % fits the width of the page to the window
    pdftitle={Grounding Language Understanding 2017},    % title
    pdfauthor={Giampiero Salvi, Stéphane Dupont},     % author
%    pdfsubject={Subject},   % subject of the document
%    pdfcreator={Creator},   % creator of the document
%    pdfproducer={Producer}, % producer of the document
    pdfkeywords={speech} {language} {grounding} {affordances}, % list of keywords
%    pdfnewwindow=true,      % links in new window
%    colorlinks=false,       % false: boxed links; true: colored links
%    linkcolor=red,          % color of internal links
%    citecolor=green,        % color of links to bibliography
%    filecolor=magenta,      % color of file links
%    urlcolor=cyan           % color of external links
    plainpages=false         % for the authorindex package
}
\def\theaipage{\string\hyperpage{\thepage}}
\isbn{XXX-XX-XXXX-XXX-X}
\howpublished{Electronic version}

\usepackage[utf8]{inputenc}
\usepackage{multicol}

\title{Grounding Language Understanding 2017}
\conferencetype{International Workshop}
\date{Aug 25, 2017}
\address{Stockholm, Sweden}

\editor{Giampiero Salvi and Stéphane Dupont}
\school{KTH Royal Institute of Technology, Stockholm, Sweden\\University of Mons, Belgium}
\issn{XXXX-XXXX}

\makeindex

\begin{document}
\pagenumbering{roman}
\maketitle
%\cleardoublepage
\newpage
\section*{Foreword}\addcontentsline{toc}{section}{Foreword}
Welcome to the First International Conference on Grounding Language Understanding (GLU 2017).

\section*{Acknowledgements}
We would like to thank Prof. Jean Rouat from Université de Sherbrooke, Canada, for his advice and help in organising this workshop. Without his contribution, the workshop would have assumed a very different form. We would also like to thank the members of the scientific committee for their help with ensuring the high standards of publication in these workshop proceedings.

\vspace{1cm}
\noindent The GLU 2017 organising committee

\noindent Stockholm, August 2017.

\newpage
\section*{Scientific Committee}\addcontentsline{toc}{section}{Referees}
\begin{itemize}
\item Leonardo Badino, Italian Institute of Technology, Italy
\item Claude Barras, LIMSI-CNRS, France
\item Tony Belpaeme, Plymouth University, UK
\item Alexandre Bernardino, Instituto Superior Técnico, Lisbon, Portugal
\item Angelo Cangelosi, Plymouth University, UK
\item Javier Civera, Universidad de Zaragoza, Spain
\item Aaron Courville, Universit de Montral, Canada
\item Laurence Devillers, LIMSI-CNRS, France
\item Stéphane Dupont, University of Mons, Belgium
\item Thierry Dutoit, University of Mons, Belgium
\item Begoña García-Zapirain, University of Deusto, Bilbao, Spain
\item Denis Jouvet, Loria, France
\item Robert Legenstein, Graz University of Technology, Austria
\item Mikołaj Leszczuk, AGH Unviersity, Krakow, Poland
\item Manuel Lopes, Instituto Superior Técnico, Lisbon, Portugal
\item Cynthia Matuszek, University of Maryland, Baltimore County, USA
\item Marie-Francine Moens, Katholieke Universiteit Leuven, Heverlee, Belgium
\item Ana C Murillo, Universidad de Zaragoza, Spain
\item Pierre-Yves Oudeyer, Inria, France
\item Michael Spranger, Sony Computer Science Laboratories Inc., Tokyo, Japan
\item Olivier Pietquin, Université de Lille, France
\item Jean Rouat, Universit de Sherbrooke, Canada
\item Marco Sabato Siniscalchi, Kore University, Enna, Italy
\item Giampiero Salvi, KTH Royal Institute of Technology, Stockholm, Sweden
\item José Santos-Victor, Instituto Superior Técnico, Lisbon, Portugal
\item Kamel Smaïli, Université de Lorraine , Nancy, France
\item Hugo Van hamme, University of Leuven, Belgium
\end{itemize}
%\cleardoublepage
\newpage
\tableofcontents

\cleardoublepage
% Keynote speakers
\pagenumbering{arabic}
\refstepcounter{section}\addcontentsline{toc}{section}{Invited papers}
\includepaper{Binding through assemblies in the human brain: Recent data and models}{Robert Legenstein}{keynotes/GLU2017_keynote_01} % fix info
%Recent experimental data has provided valuable insights into the representation of concepts and language in the human brain. Electrode recordings from the human brain suggest that concepts are represented in the medial temporal lobe (MTL) through sparse sets of neurons (assemblies). Further, fMRI recordings from the human brain suggest that specific subregions of the temporal cortex are dedicated to the representation of specific roles (e.g., subject or object) of concepts in a sentence or visually presented episode. We propose that quickly recruited assemblies of neurons in these subregions act as pointers  to previously created assemblies that represent concepts. We refer to these pointers as assembly pointers. In this computational architecture, the thematic role of a word in a sentence or episode (e.g., agent or patient) is bound to a concrete filler (e.g., a word or concept) during language processing, an operation that has been termed variable binding. We provide a proof of principle that the resulting model for binding through assembly pointers can be implemented in networks of spiking neurons, and supports basic operations of brain computations, such as structured recall and the flexible handling of information.
\includepaper{Recursion all the way: in Language, Action and Semantic Association}{Katerina Pastra}{keynotes/GLU2017_keynote_02} % fix info
% The phenomenon of recursion has been considered to be a unique characteristic of human language. However, increasing evidence in Neuroscience points to the fact that a fundamental syntactic mechanism is shared between language and action, both of which have a hieararchical and compositional organisation; Broca's area has been suggested as the neural locus of this mechanism.  In this talk, we will present the first formal specification of action with biological bases, the Minimalist Grammar of Action. The grammar allows the development of  generative computational models for action in the motor and visuomotor space. Through the grammar, we present examples of recursion in the action space and how a generative action perception/execution system may account for the phenomenon. Furthemore, we go a step further, arguing that recursion is not a phenomenon that arises in the language or action space only; it is a phenomenon that lies in any 'syntactic' activity, in any space that comprises 'merging' of elements into more and more complex units. We present recursion in the Semantic Association space (Semantic Memory), through the PRAXICON, the first ever recursive and referential semantic network. We will demonstrate the importance of the network in generalisation and reasoning for a number of applications and discuss the interdisciplinary implications of our argument on recursion. 
\includepaper{}{Emmanuel Dupoux}{keynotes/GLU2017_paper_10} % fix info
\cleardoublepage
% rest of papers
\refstepcounter{section}\addcontentsline{toc}{section}{Reviewed papers}
\includepaper{Visual Speech Speeds Up Auditory Identification Responses}{Tim Paris, Jeesun Kim, Chris Davis}{articles/avsp11_submission}
\includepaper{Do infants detect A$\rightarrow$V articulator congruency for nonnative click consonants?}{Catherine Best, Christian Kroos, Julia Irwin}{articles/avsp11_submission}
\includepaper{Perceiving Visual Prosody from Point-Light Displays}{Erin Cvejic, Jeesun Kim, Chris Davis}{articles/avsp11_submission}
\includepaper{Binding and unbinding the McGurk effect in audiovisual speech fusion: Follow-up experiments on a new paradigm}{Olha Nahorna, Fr{\'e}d{\'e}ric Berthommier, Jean-Luc Schwartz}{articles/avsp11_submission}
\includepaper{Children’s expression of uncertainty in collaborative and competitive contexts}{Mandy Visser, Emiel Krahmer, Marc Swerts}{articles/avsp11_submission}
\includepaper{The effect of seeing the interlocutor on auditory and visual speech production in noise.}{Michael Fitzpatrick, Jeesun Kim, Chris Davis}{articles/avsp11_submission}
\includepaper{Auditory-Visual Discrimination and Identification of Lexical Tone Within and Across Tone Languages}{Denis Burnham, Virginie Attina, Benjawan Kasisopa}{articles/avsp11_submission}
\includepaper{Audiovisual competition in the perception of counter-expectational questions}{Joan Borr{\`a}s-Comes, Cecilia Pugliesi, Pilar Prieto}{articles/avsp11_submission}
\includepaper{Introducing Visual Target Cost within an Acoustic-Visual Unit-Selection Speech Synthesizer}{Utpala Musti, Vincent Colotte, Asterios Toutios, Slim Ouni}{articles/avsp11_submission}
\includepaper{Auditory and Photo-realistic Audiovisual Speech Synthesis for Dutch}{Wesley Mattheyses, Lukas Latacz, Werner Verhelst}{articles/avsp11_submission}
\includepaper{Realistic Visual Speech Synthesis Based on AAM Features and an Articulatory DBN Model with Constrained Asynchrony}{Peng Wu, Dongmei Jiang, He Zhang, Hichem Sahli}{articles/avsp11_submission}
\includepaper{Talking Heads for Elderly and Alzheimer Patients (THEA): Project Report and Demonstration}{Sascha Fagel}{articles/avsp11_submission}
\includepaper{Improving Naturalness of Visual Speech Synthesis}{Laszlo Czap}{articles/avsp11_submission}
\includepaper{A robotic head using projected animated faces.}{Samer {Al Moubayed}, Simon Alexandersson, Jonas Beskow, Bj{\"o}rn Granstr{\"o}m}{articles/avsp11_submission}
\includepaper{Audiovisual speech processing in visual speech noise}{Kim Jeesun, Chris Davis}{articles/avsp11_submission}
\includepaper{Audiovisual streaming in voicing perception: new evidence for a low level interaction between audio and visual modalities}{Fr{\'e}d{\'e}ric Berthommier, Jean-Luc Schwartz}{articles/avsp11_submission}
\includepaper{An ordinal model of the McGurk illusion}{Tobias Andersen}{articles/avsp11_submission}
\includepaper{Thin slices of head movements during problem solving reveal level of difficulty}{Bart Joosten, Marije {Van Amelsvoort}, Emiel Krahmer, Eric Postma}{articles/avsp11_submission}
\includepaper{Dimensional Mapping of Multimodal Integration on Audiovisual Emotion Perception}{Yoshiko Arimoto, Kazuo Okanoya}{articles/avsp11_submission}
\includepaper{Turn-taking Control Using Gaze in Multiparty Human-Computer Dialogue: Effects of 2D and 3D Displays}{Samer {Al Moubayed}, Gabriel Skantze}{articles/avsp11_submission}
\includepaper{Bilingual Corpus for AVASR using Multiple Sensors and Depth Information}{Georgios Galatas, Gerasimos Potamianos, Dimitrios Kosmopoulos, Chris Mcmurrough, Fillia Makedon}{articles/avsp11_submission}
\includepaper{Kinetic Data for Large-Scale Analysis and Modeling of Face-to-Face Conversation}{Jonas Beskow, Simon Alexandersson, Samer {Al Moubayed}, Jens Edlund, David House}{articles/avsp11_submission}
\includepaper{``Mask-bot'' --- a life-size talking head animation robot for AV speech and human-robot communication research}{Takaaki Kuratate, Brennand Pierce, Gordon Cheng}{articles/avsp11_submission}
\includepaper{Development of Communication Support System using Lip Reading}{Takeshi Saitoh}{articles/avsp11_submission}
\includepaper{Lucia-WebGL:  a Web Based Italian MPEG-4 Talking Head}{Giuseppe Riccardo Leone, Piero Cosi}{articles/avsp11_submission}
%\includepaper{Comparison of Viseme Definitions for Visual Speech Recognition}{Luca Cappelletta, Naomi Harte}{articles/avsp11_submission}
\includepaper[0 -15]{Improved Detection of Ball Hit Events in a Tennis Game Using Multimodal Information}{Qiang Huang, Stephen Cox}{articles/avsp11_submission}
\includepaper{Speech-driven lip motion generation for tele-operated humanoid robots}{Carlos Ishi, Chaoran Liu, Hiroshi Ishiguro, Norihiro Hagita}{articles/avsp11_submission}
\includepaper{On the Audiovisual Asynchrony of Speech}{Laszlo Czap}{articles/avsp11_submission}


\section*{Author Index}\addcontentsline{toc}{section}{Author Index}
\begin{multicols}{2}
\printauthorindex
\end{multicols}
\end{document}

% do not remove (emacs configuration)
% Local variables:
% enable-local-variables: t
% ispell-local-dictionary: "british"
% mode: latex
% eval: (flyspell-mode)
% eval: (flyspell-buffer)
% End:
