% Copyright 2017, Giampiero Salvi <giampi@kth.se>
% electronic version of the proceedings
\documentclass{confproc}
\usepackage[utf8]{inputenc}
\usepackage{hyperref}
\hypersetup{
%    bookmarks=true,         % show bookmarks bar?
%    unicode=false,          % non-Latin characters in Acrobat’s bookmarks
%    pdftoolbar=true,        % show Acrobat’s toolbar?
%    pdfmenubar=true,        % show Acrobat’s menu?
%    pdffitwindow=false,     % window fit to page when opened
%    pdfstartview={FitH},    % fits the width of the page to the window
    pdftitle={Grounding Language Understanding 2017},    % title
    pdfauthor={Giampiero Salvi, Stéphane Dupont},     % author
%    pdfsubject={Subject},   % subject of the document
%    pdfcreator={Creator},   % creator of the document
%    pdfproducer={Producer}, % producer of the document
    pdfkeywords={speech} {language} {grounding} {affordances}, % list of keywords
%    pdfnewwindow=true,      % links in new window
%    colorlinks=false,       % false: boxed links; true: colored links
%    linkcolor=red,          % color of internal links
%    citecolor=green,        % color of links to bibliography
%    filecolor=magenta,      % color of file links
%    urlcolor=cyan           % color of external links
    plainpages=false         % for the authorindex package
}
\def\theaipage{\string\hyperpage{\thepage}}
\isbn{978-91-639-4916-6}
\howpublished{Electronic version}

\usepackage{multicol}

\title{Grounding Language Understanding (GLU~2017)}
\conferencetype{First International Workshop}
\date{Aug 25, 2017}
\address{Stockholm, Sweden}

\editor{Giampiero Salvi$^1$ and Stéphane Dupont$^2$}
\school{$^{1)}$ KTH Royal Institute of Technology, Stockholm, Sweden\\$^{2)}$ University of Mons, Belgium}
\publishedby{KTH Royal Institute of Technology, Stockholm, Sweden}
%\issn{1613-0073}

\makeindex

\begin{document}

\pagenumbering{roman}
\pagestyle{empty}
\maketitle
%\cleardoublepage
\newpage
\section*{Foreword}\addcontentsline{toc}{section}{Foreword}
Welcome to the First International Conference on Grounding Language Understanding (GLU~2017).

In a world where robots and intelligent systems are sharing the same environment with humans, new challenges in human-machine interaction emerge.
Problems of online grounded language acquisition, understanding and visualisation in the context of mixed-initiative human-agent interactions become essential if humans and machines are to understand each other and co-exist smoothly.
The scientific community is aware of the challenge and a number of venues have been created to discuss progress.
Noteworthy, for example, are the Google workshop on Language Grounding for Robotics that took place three weeks before GLU~2017 in Vancouver, Canada, or the 2nd Workshop on Language Learning at 2017 IEEE ICDL-EPIROB in Lisbon, Portugal in September 2017, or, finally, the Third International Workshop on Intrinsically Motivated Open-ended Learning, in Rome, Italy in October 2017.

Funding agencies are also focusing on this challenge.
An example is the CHIST-ERA call for Human Language Understanding: Grounding Language Learning that funded a number of projects including IGLU (Interactive Grounded Language Understanding), the project that promoted the GLU 2017 workshop initiative.
The GLU~2017 workshop originates from the need to share results between the projects that were funded within the CHIST-ERA HLU call, but quickly assumed a much wider scope.
Researchers from 17 countries spanning 4 continents attended the workshop.
More importantly, many different domains were included, ranging from developmental robotics, speech technology, machine learning, cognitive and systems neuroscience.

We received 21 submissions to the workshop. All submissions received at least three reviews from the Scientific Committee (in some cases four reviews were performed). Following the review suggestions and scores, 12 papers were accepted for oral presentation, 6 for poster presentation and 3 were rejected.

We were also fortunate to welcome three outstanding keynote speakers to the workshop:
\begin{description}
\item[Robert Legenstein] is Associate Professor at the Institute for Theoretical Computer Science at the Graz University of Technology, Austria and Associate Editor for IEEE Transactions on Neural Networks. His research interests range from computation and learning in networks of spiking neurons to computational complexity theory. He brought the neuroscience perspective to the representation of words in the brain.
\item[Katerina Pastra] is the director of the Cognitive Systems Research Institute, Athens, Greece. She was coordinator for a number of European projects related to the topic, such as Poeticon and Poeticon++. She brought the linguistic perspective to the problem of grounding.
\item[Emmanuel Dupoux] is director of studies at École des Hautes Etudes en Sciences Sociales, Laboratoire de Science Cognitive et Psycholinguistique, Paris, France. His focus on early acquisition of linguistic and social skills in infants brought yet another angle to the workshop discussions.
\end{description}
\section*{Acknowledgements}
We would like to thank Prof. \textbf{Jean Rouat} from Université de Sherbrooke, Canada, the coordinator of the CHIST-ERA IGLU project, for his advice and help in organising this workshop. Without his contribution, the workshop would have assumed a very different form. We would also like to thank the members of the scientific committee for their help with ensuring the high standards of publication in these workshop proceedings.

\vspace{5mm}
\noindent The GLU 2017 organising committee

\noindent Stockholm, August 2017.

\newpage
\section*{Scientific Committee}\addcontentsline{toc}{section}{Referees}
\begin{itemize}
\item Leonardo Badino, Italian Institute of Technology, Italy
\item Claude Barras, LIMSI-CNRS, France
\item Tony Belpaeme, Plymouth University, UK
\item Alexandre Bernardino, Instituto Superior Técnico, Lisbon, Portugal
\item Angelo Cangelosi, Plymouth University, UK
\item Javier Civera, Universidad de Zaragoza, Spain
\item Aaron Courville, Université de Montréal, Canada
\item Laurence Devillers, LIMSI-CNRS, France
\item Stéphane Dupont, University of Mons, Belgium
\item Thierry Dutoit, University of Mons, Belgium
\item Begoña García-Zapirain, University of Deusto, Bilbao, Spain
\item Denis Jouvet, Loria, France
\item Robert Legenstein, Graz University of Technology, Austria
\item Mikołaj Leszczuk, AGH Unviersity, Krakow, Poland
\item Manuel Lopes, Instituto Superior Técnico, Lisbon, Portugal
\item Cynthia Matuszek, University of Maryland, Baltimore County, USA
\item Marie-Francine Moens, Katholieke Universiteit Leuven, Heverlee, Belgium
\item Ana C Murillo, Universidad de Zaragoza, Spain
\item Pierre-Yves Oudeyer, Inria, France
\item Michael Spranger, Sony Computer Science Laboratories Inc., Tokyo, Japan
\item Olivier Pietquin, Université de Lille, France
\item Jean Rouat, Université de Sherbrooke, Canada
\item Marco Sabato Siniscalchi, Kore University, Enna, Italy
\item Giampiero Salvi, KTH Royal Institute of Technology, Stockholm, Sweden
\item José Santos-Victor, Instituto Superior Técnico, Lisbon, Portugal
\item Kamel Smaïli, Université de Lorraine , Nancy, France
\item Kalin Stefanov, KTH Royal Institute of Technology, Stockholm, Sweden
\item Hugo Van hamme, University of Leuven, Belgium
\end{itemize}

\end{document}
%\cleardoublepage
\newpage
\tableofcontents

\pagenumbering{arabic}
\includepaper{Visual Speech Speeds Up Auditory Identification Responses}{Tim Paris, Jeesun Kim, Chris Davis}{articles/avsp11_submission}
\includepaper{Do infants detect A$\rightarrow$V articulator congruency for nonnative click consonants?}{Catherine Best, Christian Kroos, Julia Irwin}{articles/avsp11_submission}
\includepaper{Perceiving Visual Prosody from Point-Light Displays}{Erin Cvejic, Jeesun Kim, Chris Davis}{articles/avsp11_submission}
\includepaper{Binding and unbinding the McGurk effect in audiovisual speech fusion: Follow-up experiments on a new paradigm}{Olha Nahorna, Fr{\'e}d{\'e}ric Berthommier, Jean-Luc Schwartz}{articles/avsp11_submission}
\includepaper{Children’s expression of uncertainty in collaborative and competitive contexts}{Mandy Visser, Emiel Krahmer, Marc Swerts}{articles/avsp11_submission}
\includepaper{The effect of seeing the interlocutor on auditory and visual speech production in noise.}{Michael Fitzpatrick, Jeesun Kim, Chris Davis}{articles/avsp11_submission}
\includepaper{Auditory-Visual Discrimination and Identification of Lexical Tone Within and Across Tone Languages}{Denis Burnham, Virginie Attina, Benjawan Kasisopa}{articles/avsp11_submission}
\includepaper{Audiovisual competition in the perception of counter-expectational questions}{Joan Borr{\`a}s-Comes, Cecilia Pugliesi, Pilar Prieto}{articles/avsp11_submission}
\includepaper{Introducing Visual Target Cost within an Acoustic-Visual Unit-Selection Speech Synthesizer}{Utpala Musti, Vincent Colotte, Asterios Toutios, Slim Ouni}{articles/avsp11_submission}
\includepaper{Auditory and Photo-realistic Audiovisual Speech Synthesis for Dutch}{Wesley Mattheyses, Lukas Latacz, Werner Verhelst}{articles/avsp11_submission}
\includepaper{Realistic Visual Speech Synthesis Based on AAM Features and an Articulatory DBN Model with Constrained Asynchrony}{Peng Wu, Dongmei Jiang, He Zhang, Hichem Sahli}{articles/avsp11_submission}
\includepaper{Talking Heads for Elderly and Alzheimer Patients (THEA): Project Report and Demonstration}{Sascha Fagel}{articles/avsp11_submission}
\includepaper{Improving Naturalness of Visual Speech Synthesis}{Laszlo Czap}{articles/avsp11_submission}
\includepaper{A robotic head using projected animated faces.}{Samer {Al Moubayed}, Simon Alexandersson, Jonas Beskow, Bj{\"o}rn Granstr{\"o}m}{articles/avsp11_submission}
\includepaper{Audiovisual speech processing in visual speech noise}{Kim Jeesun, Chris Davis}{articles/avsp11_submission}
\includepaper{Audiovisual streaming in voicing perception: new evidence for a low level interaction between audio and visual modalities}{Fr{\'e}d{\'e}ric Berthommier, Jean-Luc Schwartz}{articles/avsp11_submission}
\includepaper{An ordinal model of the McGurk illusion}{Tobias Andersen}{articles/avsp11_submission}
\includepaper{Thin slices of head movements during problem solving reveal level of difficulty}{Bart Joosten, Marije {Van Amelsvoort}, Emiel Krahmer, Eric Postma}{articles/avsp11_submission}
\includepaper{Dimensional Mapping of Multimodal Integration on Audiovisual Emotion Perception}{Yoshiko Arimoto, Kazuo Okanoya}{articles/avsp11_submission}
\includepaper{Turn-taking Control Using Gaze in Multiparty Human-Computer Dialogue: Effects of 2D and 3D Displays}{Samer {Al Moubayed}, Gabriel Skantze}{articles/avsp11_submission}
\includepaper{Bilingual Corpus for AVASR using Multiple Sensors and Depth Information}{Georgios Galatas, Gerasimos Potamianos, Dimitrios Kosmopoulos, Chris Mcmurrough, Fillia Makedon}{articles/avsp11_submission}
\includepaper{Kinetic Data for Large-Scale Analysis and Modeling of Face-to-Face Conversation}{Jonas Beskow, Simon Alexandersson, Samer {Al Moubayed}, Jens Edlund, David House}{articles/avsp11_submission}
\includepaper{``Mask-bot'' --- a life-size talking head animation robot for AV speech and human-robot communication research}{Takaaki Kuratate, Brennand Pierce, Gordon Cheng}{articles/avsp11_submission}
\includepaper{Development of Communication Support System using Lip Reading}{Takeshi Saitoh}{articles/avsp11_submission}
\includepaper{Lucia-WebGL:  a Web Based Italian MPEG-4 Talking Head}{Giuseppe Riccardo Leone, Piero Cosi}{articles/avsp11_submission}
%\includepaper{Comparison of Viseme Definitions for Visual Speech Recognition}{Luca Cappelletta, Naomi Harte}{articles/avsp11_submission}
\includepaper[0 -15]{Improved Detection of Ball Hit Events in a Tennis Game Using Multimodal Information}{Qiang Huang, Stephen Cox}{articles/avsp11_submission}
\includepaper{Speech-driven lip motion generation for tele-operated humanoid robots}{Carlos Ishi, Chaoran Liu, Hiroshi Ishiguro, Norihiro Hagita}{articles/avsp11_submission}
\includepaper{On the Audiovisual Asynchrony of Speech}{Laszlo Czap}{articles/avsp11_submission}


\section*{Author Index}\addcontentsline{toc}{section}{Author Index}
\begin{multicols}{2}
\printauthorindex
\end{multicols}
\end{document}

% do not remove (emacs configuration)
% Local variables:
% enable-local-variables: t
% ispell-local-dictionary: "british"
% mode: latex
% eval: (flyspell-mode)
% eval: (flyspell-buffer)
% End:
